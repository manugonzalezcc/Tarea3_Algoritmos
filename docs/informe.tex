%%%%%%%%%%%%%%%%%%%%%%%%%%%%%%%%%%%%%%%%%%%%%%%%%%%%%%%%%%%
% --------------------------------------------------------
% Rho
% LaTeX Template
% Version 2.1.1 (01/09/2024)
%
% Authors: 
% Guillermo Jimenez (memo.notess1@gmail.com)
% Eduardo Gracidas (eduardo.gracidas29@gmail.com)
% 
% License:
% Creative Commons CC BY 4.0
% --------------------------------------------------------
%%%%%%%%%%%%%%%%%%%%%%%%%%%%%%%%%%%%%%%%%%%%%%%%%%%%%%%%%%%

\documentclass[9pt,letterpaper,onecolumn]{rho-class/rho}
\usepackage[spanish,es-nodecimaldot,es-noindentfirst]{babel}
\usepackage{amsmath}
\usepackage{graphicx}
\usepackage{hyperref}

\setbool{rho-abstract}{true}
\setbool{corres-info}{true}

%----------------------------------------------------------
% TITLE
%----------------------------------------------------------

\title{}

%----------------------------------------------------------
% AUTHORS AND AFFILIATIONS
%----------------------------------------------------------

\author[$\dagger$]{Diego Sanhueza}
\author[$\dagger$]{Manuel González}
\author[$\dagger$]{Claudio Matulich}

%----------------------------------------------------------

\affil[$\dagger$]{Universidad de Magallanes}

%----------------------------------------------------------
% DATES
%----------------------------------------------------------

\dates{\today}

%----------------------------------------------------------
% FOOTER INFORMATION
%----------------------------------------------------------

\smalltitle{Diseño de Algoritmos - PR3}

%----------------------------------------------------------
% ARTICLE INFORMATION
%----------------------------------------------------------

%----------------------------------------------------------
% ABSTRACT
%----------------------------------------------------------

\begin{abstract}
%Resumen del informe
\end{abstract}

%----------------------------------------------------------

\begin{document}
\maketitle
\thispagestyle{firststyle}
\tableofcontents

%----------------------------------------------------------

\section{Introducción}
Descripción general del proyecto, motivación y contexto.

\section{Objetivos}
\begin{itemize}
    \item Implementar algoritmos avanzados de búsqueda de patrones en texto.
    \item Desarrollar estructuras de datos eficientes para indexación y recuperación de información.
    \item Analizar empíricamente la complejidad y rendimiento de los algoritmos.
    \item Aplicar los algoritmos a un problema práctico de procesamiento de documentos.
    \item Implementar técnicas de preprocesamiento de texto para mejorar la eficiencia de búsqueda.
    \item Documentar adecuadamente el código y los resultados del análisis.
\end{itemize}

\section{Descripción del Proyecto}
Explicación del problema a resolver, alcance y requisitos principales.

\section{Requisitos Técnicos}
\subsection{Algoritmos de Búsqueda de Patrones}
Descripción de los algoritmos implementados (KMP, Boyer-Moore, Shift-And, etc.), detalles de implementación y optimizaciones.

\subsection{Estructuras de Datos para Indexación}
Explicación de las estructuras usadas (Trie, índice invertido, tabla hash, filtro de Bloom, etc.).

\subsection{Técnicas de Preprocesamiento de Texto}
Descripción de las técnicas aplicadas (tokenización, normalización, stopwords, stemming, etc.).

\subsection{Framework de Análisis de Rendimiento}
Cómo se midió el rendimiento, qué métricas se usaron y cómo se generaron los datos de prueba.

\section{Desarrollo del Sistema}
\subsection{Carga y Preprocesamiento de Documentos}
Soporte para archivos de texto y HTML, extracción de texto y metadatos.

\subsection{Indexación de Documentos}
Construcción y actualización de índices, persistencia.

\subsection{Motor de Búsqueda}
Búsqueda exacta, aproximada, consultas booleanas, ranking de resultados.

\subsection{Análisis de Texto}
Estadísticas, palabras clave, similitud entre documentos.

\subsection{Interfaz de Línea de Comandos}
Descripción de los comandos, flags, formato de salida y ayuda.

\section{Pruebas y Resultados Experimentales}
Presentación de los experimentos, casos de prueba, resultados, tablas y gráficos comparativos.

\section{Discusión y Conclusiones}
Análisis de los resultados, dificultades, posibles mejoras y conclusiones finales.

\section{Referencias}
\begin{itemize}
    \item 
    \item 
    \item 
\end{itemize}

\end{document}
